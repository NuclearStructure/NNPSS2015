%%
%% Automatically generated file from DocOnce source
%% (https://github.com/hplgit/doconce/)
%%
%%


%-------------------- begin preamble ----------------------

\documentclass[%
twoside,                 % oneside: electronic viewing, twoside: printing
final,                   % or draft (marks overfull hboxes, figures with paths)
10pt]{article}

\listfiles               % print all files needed to compile this document

\usepackage{relsize,makeidx,color,setspace,amsmath,amsfonts}
\usepackage[table]{xcolor}
\usepackage{bm,microtype}

\usepackage[T1]{fontenc}
%\usepackage[latin1]{inputenc}
\usepackage{ucs}
\usepackage[utf8x]{inputenc}

\usepackage{lmodern}         % Latin Modern fonts derived from Computer Modern

% Hyperlinks in PDF:
\definecolor{linkcolor}{rgb}{0,0,0.4}
\usepackage{hyperref}
\hypersetup{
    breaklinks=true,
    colorlinks=true,
    linkcolor=linkcolor,
    urlcolor=linkcolor,
    citecolor=black,
    filecolor=black,
    %filecolor=blue,
    pdfmenubar=true,
    pdftoolbar=true,
    bookmarksdepth=3   % Uncomment (and tweak) for PDF bookmarks with more levels than the TOC
    }
%\hyperbaseurl{}   % hyperlinks are relative to this root

\setcounter{tocdepth}{2}  % number chapter, section, subsection

% prevent orhpans and widows
\clubpenalty = 10000
\widowpenalty = 10000

% --- end of standard preamble for documents ---


% insert custom LaTeX commands...

\raggedbottom
\makeindex

%-------------------- end preamble ----------------------

\begin{document}

% endif for #ifdef PREAMBLE


% ------------------- main content ----------------------

% Slides for PHY981


% ----------------- title -------------------------

\thispagestyle{empty}

\begin{center}
{\LARGE\bf
\begin{spacing}{1.25}
Many-body perturbation theory
\end{spacing}
}
\end{center}

% ----------------- author(s) -------------------------

\begin{center}
{\bf \href{{http://computationalphysics.no}}{Morten Hjorth-Jensen}, National Superconducting Cyclotron Laboratory and Department of Physics and Astronomy, Michigan State University, East Lansing, MI 48824, USA {\&} Department of Physics, University of Oslo, Oslo, Norway${}^{}$} \\ [0mm]
\end{center}

    \begin{center}
% List of all institutions:
\end{center}
    
% ----------------- end author(s) -------------------------

\begin{center} % date
June 15-26 2015
\end{center}

\vspace{1cm}


\tableofcontents


\vspace{1cm} % after toc




\subsection*{Many-body perturbation theory}

We assume here that we are only interested in the ground state of the system and 
expand the exact wave function in term of a series of Slater determinants
\[
\vert \Psi_0\rangle = \vert \Phi_0\rangle + \sum_{m=1}^{\infty}C_m\vert \Phi_m\rangle,
\]
where we have assumed that the true ground state is dominated by the 
solution of the unperturbed problem, that is
\[
\hat{H}_0\vert \Phi_0\rangle= W_0\vert \Phi_0\rangle.
\]
The state $\vert \Psi_0\rangle$ is not normalized, rather we have used an intermediate 
normalization $\langle \Phi_0 \vert \Psi_0\rangle=1$ since we have $\langle \Phi_0\vert \Phi_0\rangle=1$. 



The Schroedinger equation is
\[
\hat{H}\vert \Psi_0\rangle = E\vert \Psi_0\rangle,
\]
and multiplying the latter from the left with $\langle \Phi_0\vert $ gives
\[
\langle \Phi_0\vert \hat{H}\vert \Psi_0\rangle = E\langle \Phi_0\vert \Psi_0\rangle=E,
\]
and subtracting from this equation
\[
\langle \Psi_0\vert \hat{H}_0\vert \Phi_0\rangle= W_0\langle \Psi_0\vert \Phi_0\rangle=W_0,
\]
and using the fact that the both operators $\hat{H}$ and $\hat{H}_0$ are hermitian 
results in
\[
\Delta E=E-W_0=\langle \Phi_0\vert \hat{H}_I\vert \Psi_0\rangle,
\]
which is an exact result. We call this quantity the correlation energy.



This equation forms the starting point for all perturbative derivations. However,
as it stands it represents nothing but a mere formal rewriting of Schroedinger's equation and is not of much practical use. The exact wave function $\vert \Psi_0\rangle$ is unknown. In order to obtain a perturbative expansion, we need to expand the exact wave function in terms of the interaction $\hat{H}_I$. 

Here we have assumed that our model space defined by the operator $\hat{P}$ is one-dimensional, meaning that
\[
\hat{P}= \vert \Phi_0\rangle \langle \Phi_0\vert ,
\]
and
\[
\hat{Q}=\sum_{m=1}^{\infty}\vert \Phi_m\rangle \langle \Phi_m\vert .
\]


We can thus rewrite the exact wave function as
\[
\vert \Psi_0\rangle= (\hat{P}+\hat{Q})\vert \Psi_0\rangle=\vert \Phi_0\rangle+\hat{Q}\vert \Psi_0\rangle.
\]
Going back to the Schr\"odinger equation, we can rewrite it as, adding and a subtracting a term $\omega \vert \Psi_0\rangle$ as
\[
\left(\omega-\hat{H}_0\right)\vert \Psi_0\rangle=\left(\omega-E+\hat{H}_I\right)\vert \Psi_0\rangle,
\]
where $\omega$ is an energy variable to be specified later. 


We assume also that the resolvent of $\left(\omega-\hat{H}_0\right)$ exits, that is
it has an inverse which defined the unperturbed Green's function as
\[
\left(\omega-\hat{H}_0\right)^{-1}=\frac{1}{\left(\omega-\hat{H}_0\right)}.
\]

We can rewrite Schroedinger's equation as
\[
\vert \Psi_0\rangle=\frac{1}{\omega-\hat{H}_0}\left(\omega-E+\hat{H}_I\right)\vert \Psi_0\rangle,
\]
and multiplying from the left with $\hat{Q}$ results in
\[
\hat{Q}\vert \Psi_0\rangle=\frac{\hat{Q}}{\omega-\hat{H}_0}\left(\omega-E+\hat{H}_I\right)\vert \Psi_0\rangle,
\]
which is possible since we have defined the operator $\hat{Q}$ in terms of the eigenfunctions of $\hat{H}$.




These operators commute meaning that
\[
\hat{Q}\frac{1}{\left(\omega-\hat{H}_0\right)}\hat{Q}=\hat{Q}\frac{1}{\left(\omega-\hat{H}_0\right)}=\frac{\hat{Q}}{\left(\omega-\hat{H}_0\right)}.
\]
With these definitions we can in turn define the wave function as 
\[
\vert \Psi_0\rangle=\vert \Phi_0\rangle+\frac{\hat{Q}}{\omega-\hat{H}_0}\left(\omega-E+\hat{H}_I\right)\vert \Psi_0\rangle.
\]
This equation is again nothing but a formal rewrite of Schr\"odinger's equation
and does not represent a practical calculational scheme.  
It is a non-linear equation in two unknown quantities, the energy $E$ and the exact
wave function $\vert \Psi_0\rangle$. We can however start with a guess for $\vert \Psi_0\rangle$ on the right hand side of the last equation.



 The most common choice is to start with the function which is expected to exhibit the largest overlap with the wave function we are searching after, namely $\vert \Phi_0\rangle$. This can again be inserted in the solution for $\vert \Psi_0\rangle$ in an iterative fashion and if we continue along these lines we end up with
\[
\vert \Psi_0\rangle=\sum_{i=0}^{\infty}\left\{\frac{\hat{Q}}{\omega-\hat{H}_0}\left(\omega-E+\hat{H}_I\right)\right\}^i\vert \Phi_0\rangle, 
\]
for the wave function and
\[
\Delta E=\sum_{i=0}^{\infty}\langle \Phi_0\vert \hat{H}_I\left\{\frac{\hat{Q}}{\omega-\hat{H}_0}\left(\omega-E+\hat{H}_I\right)\right\}^i\vert \Phi_0\rangle, 
\]
which is now  a perturbative expansion of the exact energy in terms of the interaction
$\hat{H}_I$ and the unperturbed wave function $\vert \Psi_0\rangle$.



In our equations for $\vert \Psi_0\rangle$ and $\Delta E$ in terms of the unperturbed
solutions $\vert \Phi_i\rangle$  we have still an undetermined parameter $\omega$
and a dependecy on the exact energy $E$. Not much has been gained thus from a practical computational point of view. 

In Brilluoin-Wigner perturbation theory it is customary to set $\omega=E$. This results in the following perturbative expansion for the energy $\Delta E$
\[
\Delta E=\sum_{i=0}^{\infty}\langle \Phi_0\vert \hat{H}_I\left\{\frac{\hat{Q}}{\omega-\hat{H}_0}\left(\omega-E+\hat{H}_I\right)\right\}^i\vert \Phi_0\rangle=
\]
\[
\langle \Phi_0\vert \left(\hat{H}_I+\hat{H}_I\frac{\hat{Q}}{E-\hat{H}_0}\hat{H}_I+
\hat{H}_I\frac{\hat{Q}}{E-\hat{H}_0}\hat{H}_I\frac{\hat{Q}}{E-\hat{H}_0}\hat{H}_I+\dots\right)\vert \Phi_0\rangle. 
\]

\[
\Delta E=\sum_{i=0}^{\infty}\langle \Phi_0\vert \hat{H}_I\left\{\frac{\hat{Q}}{\omega-\hat{H}_0}\left(\omega-E+\hat{H}_I\right)\right\}^i\vert \Phi_0\rangle=\]
\[
\langle \Phi_0\vert \left(\hat{H}_I+\hat{H}_I\frac{\hat{Q}}{E-\hat{H}_0}\hat{H}_I+
\hat{H}_I\frac{\hat{Q}}{E-\hat{H}_0}\hat{H}_I\frac{\hat{Q}}{E-\hat{H}_0}\hat{H}_I+\dots\right)\vert \Phi_0\rangle. 
\]
This expression depends however on the exact energy $E$ and is again not very convenient from a practical point of view. It can obviously be solved iteratively, by starting with a guess for  $E$ and then solve till some kind of self-consistency criterion has been reached. 

Actually, the above expression is nothing but a rewrite again of the full Schr\"odinger equation. 

Defining $e=E-\hat{H}_0$ and recalling that $\hat{H}_0$ commutes with 
$\hat{Q}$ by construction and that $\hat{Q}$ is an idempotent operator
$\hat{Q}^2=\hat{Q}$. 
Using this equation in the above expansion for $\Delta E$ we can write the denominator 
\[
\hat{Q}\frac{1}{\hat{e}-\hat{Q}\hat{H}_I\hat{Q}}=
\]
\[
\hat{Q}\left[\frac{1}{\hat{e}}+\frac{1}{\hat{e}}\hat{Q}\hat{H}_I\hat{Q}
\frac{1}{\hat{e}}+\frac{1}{\hat{e}}\hat{Q}\hat{H}_I\hat{Q}
\frac{1}{\hat{e}}\hat{Q}\hat{H}_I\hat{Q}\frac{1}{\hat{e}}+\dots\right]\hat{Q}.
\]

Inserted in the expression for $\Delta E$ leads to 
\[
\Delta E=
\langle \Phi_0\vert \hat{H}_I+\hat{H}_I\hat{Q}\frac{1}{E-\hat{H}_0-\hat{Q}\hat{H}_I\hat{Q}}\hat{Q}\hat{H}_I\vert \Phi_0\rangle. 
\]
In RS perturbation theory we set $\omega = W_0$ and obtain the following expression for the energy difference
\[
\Delta E=\sum_{i=0}^{\infty}\langle \Phi_0\vert \hat{H}_I\left\{\frac{\hat{Q}}{W_0-\hat{H}_0}\left(\hat{H}_I-\Delta E\right)\right\}^i\vert \Phi_0\rangle=
\]
\[
\langle \Phi_0\vert \left(\hat{H}_I+\hat{H}_I\frac{\hat{Q}}{W_0-\hat{H}_0}(\hat{H}_I-\Delta E)+
\hat{H}_I\frac{\hat{Q}}{W_0-\hat{H}_0}(\hat{H}_I-\Delta E)\frac{\hat{Q}}{W_0-\hat{H}_0}(\hat{H}_I-\Delta E)+\dots\right)\vert \Phi_0\rangle.
\]



Recalling that $\hat{Q}$ commutes with $\hat{H_0}$ and since $\Delta E$ is a constant we obtain that
\[
\hat{Q}\Delta E\vert \Phi_0\rangle = \hat{Q}\Delta E\vert \hat{Q}\Phi_0\rangle = 0.
\]
Inserting this results in the expression for the energy results in
\[
\Delta E=\langle \Phi_0\vert \left(\hat{H}_I+\hat{H}_I\frac{\hat{Q}}{W_0-\hat{H}_0}\hat{H}_I+
\hat{H}_I\frac{\hat{Q}}{W_0-\hat{H}_0}(\hat{H}_I-\Delta E)\frac{\hat{Q}}{W_0-\hat{H}_0}\hat{H}_I+\dots\right)\vert \Phi_0\rangle.
\]



We can now this expression in terms of a perturbative expression in terms
of $\hat{H}_I$ where we iterate the last expression in terms of $\Delta E$
\[
\Delta E=\sum_{i=1}^{\infty}\Delta E^{(i)}.
\]
We get the following expression for $\Delta E^{(i)}$
\[
\Delta E^{(1)}=\langle \Phi_0\vert \hat{H}_I\vert \Phi_0\rangle,
\] 
which is just the contribution to first order in perturbation theory,
\[
\Delta E^{(2)}=\langle\Phi_0\vert \hat{H}_I\frac{\hat{Q}}{W_0-\hat{H}_0}\hat{H}_I\vert \Phi_0\rangle, 
\]
which is the contribution to second order.



\[
\Delta E^{(3)}=\langle \Phi_0\vert \hat{H}_I\frac{\hat{Q}}{W_0-\hat{H}_0}\hat{H}_I\frac{\hat{Q}}{W_0-\hat{H}_0}\hat{H}_I\Phi_0\rangle-
\langle\Phi_0\vert \hat{H}_I\frac{\hat{Q}}{W_0-\hat{H}_0}\langle \Phi_0\vert \hat{H}_I\vert \Phi_0\rangle\frac{\hat{Q}}{W_0-\hat{H}_0}\hat{H}_I\vert \Phi_0\rangle,
\]
being the third-order contribution. 


\subsection*{Interpreting the correlation energy and the wave operator}
\begin{itemize}
\item TO DO: add material about the relation between the correlated wave operator and FCI and HF calculations
\end{itemize}

\noindent

% ------------------- end of main content ---------------


\printindex

\end{document}

