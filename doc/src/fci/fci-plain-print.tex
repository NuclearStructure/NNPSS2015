%%
%% Automatically generated file from DocOnce source
%% (https://github.com/hplgit/doconce/)
%%
%%


%-------------------- begin preamble ----------------------

\documentclass[%
twoside,                 % oneside: electronic viewing, twoside: printing
final,                   % or draft (marks overfull hboxes, figures with paths)
10pt]{article}

\listfiles               % print all files needed to compile this document

\usepackage{relsize,makeidx,color,setspace,amsmath,amsfonts}
\usepackage[table]{xcolor}
\usepackage{bm,microtype}

\usepackage[T1]{fontenc}
%\usepackage[latin1]{inputenc}
\usepackage{ucs}
\usepackage[utf8x]{inputenc}

\usepackage{lmodern}         % Latin Modern fonts derived from Computer Modern

% Hyperlinks in PDF:
\definecolor{linkcolor}{rgb}{0,0,0.4}
\usepackage{hyperref}
\hypersetup{
    breaklinks=true,
    colorlinks=true,
    linkcolor=linkcolor,
    urlcolor=linkcolor,
    citecolor=black,
    filecolor=black,
    %filecolor=blue,
    pdfmenubar=true,
    pdftoolbar=true,
    bookmarksdepth=3   % Uncomment (and tweak) for PDF bookmarks with more levels than the TOC
    }
%\hyperbaseurl{}   % hyperlinks are relative to this root

\setcounter{tocdepth}{2}  % number chapter, section, subsection

% prevent orhpans and widows
\clubpenalty = 10000
\widowpenalty = 10000

% --- end of standard preamble for documents ---


% insert custom LaTeX commands...

\raggedbottom
\makeindex

%-------------------- end preamble ----------------------

\begin{document}

% endif for #ifdef PREAMBLE


% ------------------- main content ----------------------

% Slides for PHY981


% ----------------- title -------------------------

\thispagestyle{empty}

\begin{center}
{\LARGE\bf
\begin{spacing}{1.25}
Nuclear Shell Model
\end{spacing}
}
\end{center}

% ----------------- author(s) -------------------------

\begin{center}
{\bf \href{{http://computationalphysics.no}}{Morten Hjorth-Jensen}, National Superconducting Cyclotron Laboratory and Department of Physics and Astronomy, Michigan State University, East Lansing, MI 48824, USA {\&} Department of Physics, University of Oslo, Oslo, Norway${}^{}$} \\ [0mm]
\end{center}

    \begin{center}
% List of all institutions:
\end{center}
    
% ----------------- end author(s) -------------------------

\begin{center} % date
June 15-26 2015
\end{center}

\vspace{1cm}


\tableofcontents


\vspace{1cm} % after toc




\subsection*{Slater determinants as basis states}

The simplest possible choice for many-body wavefunctions are \textbf{product} wavefunctions.
That is
\[ 
\Psi(x_1, x_2, x_3, \ldots, x_A) \approx \phi_1(x_1) \phi_2(x_2) \phi_3(x_3) \ldots
\]
because we are really only good  at thinking about one particle at a time. Such 
product wavefunctions, without correlations, are easy to 
work with; for example, if the single-particle states $\phi_i(x)$ are orthonormal, then 
the product wavefunctions are easy to orthonormalize.   

Similarly, computing matrix elements of operators are relatively easy, because the 
integrals factorize.


The price we pay is the lack of correlations, which we must build up by using many, many product 
wavefunctions.


Because we have fermions, we are required to have antisymmetric wavefunctions, that is
\[
\Psi(x_1, x_2, x_3, \ldots, x_A) = - \Psi(x_2, x_1, x_3, \ldots, x_A)
\]
etc. This is accomplished formally by using the determinantal formalism
\[
\Psi(x_1, x_2, \ldots, x_A) 
= \frac{1}{\sqrt{A!}} 
\det \left | 
\begin{array}{cccc}
\phi_1(x_1) & \phi_1(x_2) & \ldots & \phi_1(x_A) \\
\phi_2(x_1) & \phi_2(x_2) & \ldots & \phi_2(x_A) \\
 \vdots & & &  \\
\phi_A(x_1) & \phi_A(x_2) & \ldots & \phi_A(x_A) 
\end{array}
\right |
\]
Product wavefunction + antisymmetry (Pauli principle) = Slater determinant. 


Properties of the determinant (interchange of any two rows or 
any two columns yields a change in sign; thus no two rows and no 
two columns can be the same) lead to the following consequence of the Pauli principle:

\begin{itemize}
\item No two particles can be at the same place (two columns the same); and

\item No two particles can be in the same state (two rows the same).
\end{itemize}

\noindent
As a practical matter, however, Slater determinants beyond $N=4$ quickly become 
unwieldy. Thus we turn to the \textbf{occupation representation} or \textbf{second quantization} to simplify calculations. 

The occupation representation, using fermion \textbf{creation} and \textbf{annihilation} 
operators, is compact and efficient. It is also abstract and, at first encounter, not easy to 
internalize. It is inspired by other operator formalism, such as the ladder operators for 
the harmonic oscillator or for angular momentum, but unlike those cases, the operators \textbf{do not have coordinate space representations}.

Instead, one can think of fermion creation/annihilation operators as a game of symbols that 
compactly reproduces what one would do, albeit clumsily, with full coordinate-space Slater 
determinants. 



We start with a set of orthonormal single-particle states $\{ \phi_i(x) \}$. 
(Note: this requirement, and others, can be relaxed, but leads to a 
more involved formalism.) \textbf{Any} orthonormal set will do. 

To each single-particle state $\phi_i(x)$ we associate a creation operator 
$\hat{a}^\dagger_i$ and an annihilation operator $\hat{a}_i$. 

When acting on the vacuum state $| 0 \rangle$, the creation operator $\hat{a}^\dagger_i$ causes 
a particle to occupy the single-particle state $\phi_i(x)$:
\[
\phi_i(x) \rightarrow \hat{a}^\dagger_i |0 \rangle
\]



But with multiple creation operators we can occupy multiple states:
\[
\phi_i(x) \phi_j(x^\prime) \phi_k(x^{\prime \prime}) 
\rightarrow \hat{a}^\dagger_i \hat{a}^\dagger_j \hat{a}^\dagger_k |0 \rangle.
\]

Now we impose antisymmetry, by having the fermion operators satisfy  \textbf{anticommutation relations}:
\[
\hat{a}^\dagger_i \hat{a}^\dagger_j + \hat{a}^\dagger_j \hat{a}^\dagger_i
= [ \hat{a}^\dagger_i ,\hat{a}^\dagger_j ]_+ 
= \{ \hat{a}^\dagger_i ,\hat{a}^\dagger_j \} = 0
\]
so that 
\[
\hat{a}^\dagger_i \hat{a}^\dagger_j = - \hat{a}^\dagger_j \hat{a}^\dagger_i
\]




Because of this property, automatically $\hat{a}^\dagger_i \hat{a}^\dagger_i = 0$, 
enforcing the Pauli exclusion principle.  Thus when writing a Slater determinant 
using creation operators, 
\[
\hat{a}^\dagger_i \hat{a}^\dagger_j \hat{a}^\dagger_k \ldots |0 \rangle
\]
each index $i,j,k, \ldots$ must be unique.




\subsection*{Full Configuration Interaction Theory}

We have defined the ansatz for the ground state as 
\[
|\Phi_0\rangle = \left(\prod_{i\le F}\hat{a}_{i}^{\dagger}\right)|0\rangle,
\]
where the index $i$ defines different single-particle states up to the Fermi level. We have assumed that we have $N$ fermions. 
A given one-particle-one-hole ($1p1h$) state can be written as
\[
|\Phi_i^a\rangle = \hat{a}_{a}^{\dagger}\hat{a}_i|\Phi_0\rangle,
\]
while a $2p2h$ state can be written as
\[
|\Phi_{ij}^{ab}\rangle = \hat{a}_{a}^{\dagger}\hat{a}_{b}^{\dagger}\hat{a}_j\hat{a}_i|\Phi_0\rangle,
\]
and a general $ApAh$ state as 
\[
|\Phi_{ijk\dots}^{abc\dots}\rangle = \hat{a}_{a}^{\dagger}\hat{a}_{b}^{\dagger}\hat{a}_{c}^{\dagger}\dots\hat{a}_k\hat{a}_j\hat{a}_i|\Phi_0\rangle.
\]

We use letters $ijkl\dots$ for states below the Fermi level and $abcd\dots$ for states above the Fermi level. A general single-particle state is given by letters $pqrs\dots$.

We can then expand our exact state function for the ground state 
as
\[
|\Psi_0\rangle=C_0|\Phi_0\rangle+\sum_{ai}C_i^a|\Phi_i^a\rangle+\sum_{abij}C_{ij}^{ab}|\Phi_{ij}^{ab}\rangle+\dots
=(C_0+\hat{C})|\Phi_0\rangle,
\]
where we have introduced the so-called correlation operator 
\[
\hat{C}=\sum_{ai}C_i^a\hat{a}_{a}^{\dagger}\hat{a}_i  +\sum_{abij}C_{ij}^{ab}\hat{a}_{a}^{\dagger}\hat{a}_{b}^{\dagger}\hat{a}_j\hat{a}_i+\dots
\]
Since the normalization of $\Psi_0$ is at our disposal and since $C_0$ is by hypothesis non-zero, we may arbitrarily set $C_0=1$ with 
corresponding proportional changes in all other coefficients. Using this so-called intermediate normalization we have
\[
\langle \Psi_0 | \Phi_0 \rangle = \langle \Phi_0 | \Phi_0 \rangle = 1, 
\]
resulting in 
\[
|\Psi_0\rangle=(1+\hat{C})|\Phi_0\rangle.
\]


We rewrite 
\[
|\Psi_0\rangle=C_0|\Phi_0\rangle+\sum_{ai}C_i^a|\Phi_i^a\rangle+\sum_{abij}C_{ij}^{ab}|\Phi_{ij}^{ab}\rangle+\dots,
\]
in a more compact form as 
\[
|\Psi_0\rangle=\sum_{PH}C_H^P\Phi_H^P=\left(\sum_{PH}C_H^P\hat{A}_H^P\right)|\Phi_0\rangle,
\]
where $H$ stands for $0,1,\dots,n$ hole states and $P$ for $0,1,\dots,n$ particle states. 
Our requirement of unit normalization gives
\[
\langle \Psi_0 | \Psi_0 \rangle = \sum_{PH}|C_H^P|^2= 1,
\]
and the energy can be written as 
\[
E= \langle \Psi_0 | \hat{H} |\Psi_0 \rangle= \sum_{PP'HH'}C_H^{*P}\langle \Phi_H^P | \hat{H} |\Phi_{H'}^{P'} \rangle C_{H'}^{P'}.
\]


Normally 
\[
E= \langle \Psi_0 | \hat{H} |\Psi_0 \rangle= \sum_{PP'HH'}C_H^{*P}\langle \Phi_H^P | \hat{H} |\Phi_{H'}^{P'} \rangle C_{H'}^{P'},
\]
is solved by diagonalization setting up the Hamiltonian matrix defined by the basis of all possible Slater determinants. A diagonalization
is equivalent to finding the variational minimum   of 
\[
 \langle \Psi_0 | \hat{H} |\Psi_0 \rangle-\lambda \langle \Psi_0 |\Psi_0 \rangle,
\]
where $\lambda$ is a variational multiplier to be identified with the energy of the system.

The minimization process results in 
\[
\delta\left[ \langle \Psi_0 | \hat{H} |\Psi_0 \rangle-\lambda \langle \Psi_0 |\Psi_0 \rangle\right]=
\]
\[
\sum_{P'H'}\left\{\delta[C_H^{*P}]\langle \Phi_H^P | \hat{H} |\Phi_{H'}^{P'} \rangle C_{H'}^{P'}+
C_H^{*P}\langle \Phi_H^P | \hat{H} |\Phi_{H'}^{P'} \rangle \delta[C_{H'}^{P'}]-
\lambda( \delta[C_H^{*P}]C_{H'}^{P'}+C_H^{*P}\delta[C_{H'}^{P'}]\right\} = 0.
\]
Since the coefficients $\delta[C_H^{*P}]$ and $\delta[C_{H'}^{P'}]$ are complex conjugates it is necessary and sufficient to require the quantities that multiply with $\delta[C_H^{*P}]$ to vanish.  

This leads to 
\[
\sum_{P'H'}\langle \Phi_H^P | \hat{H} |\Phi_{H'}^{P'} \rangle C_{H'}^{P'}-\lambda C_H^{P}=0,
\]
for all sets of $P$ and $H$.

If we then multiply by the corresponding $C_H^{*P}$ and sum over $PH$ we obtain
\[ 
\sum_{PP'HH'}C_H^{*P}\langle \Phi_H^P | \hat{H} |\Phi_{H'}^{P'} \rangle C_{H'}^{P'}-\lambda\sum_{PH}|C_H^P|^2=0,
\]
leading to the identification $\lambda = E$. This means that we have for all $PH$ sets
\begin{equation}
\sum_{P'H'}\langle \Phi_H^P | \hat{H} -E|\Phi_{H'}^{P'} \rangle = 0. \label{eq:fullci}
\end{equation}



An alternative way to derive the last equation is to start from 
\[
(\hat{H} -E)|\Psi_0\rangle = (\hat{H} -E)\sum_{P'H'}C_{H'}^{P'}|\Phi_{H'}^{P'} \rangle=0, 
\]
and if this equation is successively projected against all $\Phi_H^P$ in the expansion of $\Psi$, we end up with Eq.~(\ref{eq:fullci}).

One solves this equation normally by diagonalization. If we are able to solve this equation exactly (that is
numerically exactly) in a large Hilbert space (it will be truncated in terms of the number of single-particle states included in the definition
of Slater determinants), it can then serve as a benchmark for other many-body methods which approximate the correlation operator
$\hat{C}$.  


\subsection*{Example of a Hamiltonian matrix}

Suppose, as an example, that we have six fermions below the Fermi level.
This means that we can make at most $6p-6h$ excitations. If we have an infinity of single particle states above the Fermi level, we will obviously have an infinity of say $2p-2h$ excitations. Each such way to configure the particles is called a \textbf{configuration}. We will always have to truncate in the basis of single-particle states.
This gives us a finite number of possible Slater determinants. Our Hamiltonian matrix would then look like (where each block can have a large dimensionalities):


\begin{quote}
\begin{tabular}{cccccccc}
\hline
\multicolumn{1}{c}{  } & \multicolumn{1}{c}{ $0p-0h$ } & \multicolumn{1}{c}{ $1p-1h$ } & \multicolumn{1}{c}{ $2p-2h$ } & \multicolumn{1}{c}{ $3p-3h$ } & \multicolumn{1}{c}{ $4p-4h$ } & \multicolumn{1}{c}{ $5p-5h$ } & \multicolumn{1}{c}{ $6p-6h$ } \\
\hline
$0p-0h$ & x       & x       & x       & 0       & 0       & 0       & 0       \\
$1p-1h$ & x       & x       & x       & x       & 0       & 0       & 0       \\
$2p-2h$ & x       & x       & x       & x       & x       & 0       & 0       \\
$3p-3h$ & 0       & x       & x       & x       & x       & x       & 0       \\
$4p-4h$ & 0       & 0       & x       & x       & x       & x       & x       \\
$5p-5h$ & 0       & 0       & 0       & x       & x       & x       & x       \\
$6p-6h$ & 0       & 0       & 0       & 0       & x       & x       & x       \\
\hline
\end{tabular}
\end{quote}

\noindent
with a two-body force. Why are there non-zero blocks of elements? 
If we use a Hartree-Fock basis, this corresponds to a particular unitary transformation where matrix elements of the type $\langle 0p-0h \vert \hat{H} \vert 1p-1h\rangle =\langle \Phi_0 | \hat{H}|\Phi_{i}^{a}\rangle=0$ and our Hamiltonian matrix becomes 


\begin{quote}
\begin{tabular}{cccccccc}
\hline
\multicolumn{1}{c}{  } & \multicolumn{1}{c}{ $0p-0h$ } & \multicolumn{1}{c}{ $1p-1h$ } & \multicolumn{1}{c}{ $2p-2h$ } & \multicolumn{1}{c}{ $3p-3h$ } & \multicolumn{1}{c}{ $4p-4h$ } & \multicolumn{1}{c}{ $5p-5h$ } & \multicolumn{1}{c}{ $6p-6h$ } \\
\hline
$0p-0h$ & $\tilde{x}$ & 0           & $\tilde{x}$ & 0           & 0           & 0           & 0           \\
$1p-1h$ & 0           & $\tilde{x}$ & $\tilde{x}$ & $\tilde{x}$ & 0           & 0           & 0           \\
$2p-2h$ & $\tilde{x}$ & $\tilde{x}$ & $\tilde{x}$ & $\tilde{x}$ & $\tilde{x}$ & 0           & 0           \\
$3p-3h$ & 0           & $\tilde{x}$ & $\tilde{x}$ & $\tilde{x}$ & $\tilde{x}$ & $\tilde{x}$ & 0           \\
$4p-4h$ & 0           & 0           & $\tilde{x}$ & $\tilde{x}$ & $\tilde{x}$ & $\tilde{x}$ & $\tilde{x}$ \\
$5p-5h$ & 0           & 0           & 0           & $\tilde{x}$ & $\tilde{x}$ & $\tilde{x}$ & $\tilde{x}$ \\
$6p-6h$ & 0           & 0           & 0           & 0           & $\tilde{x}$ & $\tilde{x}$ & $\tilde{x}$ \\
\hline
\end{tabular}
\end{quote}

\noindent
If we do not make any truncations in the possible sets of Slater determinants (many-body states) we can make by distributing $A$ nucleons among $n$ single-particle states, we call such a calculation for 
\begin{itemize}
\item Full configuration interaction theory
\end{itemize}

\noindent
If we make truncations, we have different possibilities

\begin{itemize}
\item The standard nuclear shell-model. Here we define an effective Hilbert space with respect to a given core. The calculations are normally then performed for all many-body states that can be constructed from the effective Hilbert spaces. This approach requires a properly defined effective Hamiltonian

\item We can truncate in the number of excitations. For example, we can limit the possible Slater determinants to only $1p-1h$ and $2p-2h$ excitations. This is called a configuration interaction calculation at the level of singles and doubles excitations, or just CISD. 

\item We can limit the number of excitations in terms of the excitation energies. If we do not define a core, this defines normally what is called the no-core shell-model approach. 
\end{itemize}

\noindent
What happens if we have a three-body interaction and a Hartree-Fock basis? 

Full configuration interaction theory calculations provide in principle, if we can diagonalize numerically, all states of interest. The dimensionality of the problem explodes however quickly.

The total number of Slater determinants which can be built with say $N$ neutrons distributed among $n$ single particle states is
\[
\left (\begin{array}{c} n \\ N\end{array} \right) =\frac{n!}{(n-N)!N!}. 
\]

For a model space which comprises the first for major shells only $0s$, $0p$, $1s0d$ and $1p0f$ we have $40$ single particle states for neutrons and protons.  For the eight neutrons of oxygen-16 we would then have
\[
\left (\begin{array}{c} 40 \\ 8\end{array} \right) =\frac{40!}{(32)!8!}\sim 10^{9}, 
\]
and multiplying this with the number of proton Slater determinants we end up with approximately witha dimensionality $d$ of $d\sim 10^{18}$.


This number can be reduced if we look at specific symmetries only. However, the dimensionality explodes quickly!

\begin{itemize}
\item For Hamiltonian matrices of dimensionalities  which are smaller than $d\sim 10^5$, we would use so-called direct methods for diagonalizing the Hamiltonian matrix

\item For larger dimensionalities iterative eigenvalue solvers like Lanczos' method are used. The most efficient codes at present can handle matrices of $d\sim 10^{10}$. 
\end{itemize}

\noindent
\subsection*{A non-practical way of solving the eigenvalue problem}

For reasons to come (links with Coupled-Cluster theory and Many-Body perturbation theory), 
we will rewrite Eq.~(\ref{eq:fullci}) as a set of coupled non-linear equations in terms of the unknown coefficients $C_H^P$. 
To obtain the eigenstates and eigenvalues in terms of non-linear equations is not a very practical approach. However, it serves the scope of linking FCI theory with approximative solutions to the many-body problem.

To see this, we look at the contributions arising from 
\[
\langle \Phi_H^P | = \langle \Phi_0|
\]
in  Eq.~(\ref{eq:fullci}), that is we multiply with $\langle \Phi_0 |$
from the left in 
\[
(\hat{H} -E)\sum_{P'H'}C_{H'}^{P'}|\Phi_{H'}^{P'} \rangle=0. 
\]
If we assume that we have a two-body operator at most, Slater's rule gives then an equation for the 
correlation energy in terms of $C_i^a$ and $C_{ij}^{ab}$ only.  We get then
\[
\langle \Phi_0 | \hat{H} -E| \Phi_0\rangle + \sum_{ai}\langle \Phi_0 | \hat{H} -E|\Phi_{i}^{a} \rangle C_{i}^{a}+
\sum_{abij}\langle \Phi_0 | \hat{H} -E|\Phi_{ij}^{ab} \rangle C_{ij}^{ab}=0,
\]
or 
\[
E-E_0 =\Delta E=\sum_{ai}\langle \Phi_0 | \hat{H}|\Phi_{i}^{a} \rangle C_{i}^{a}+
\sum_{abij}\langle \Phi_0 | \hat{H}|\Phi_{ij}^{ab} \rangle C_{ij}^{ab},
\]
where the energy $E_0$ is the reference energy and $\Delta E$ defines the so-called correlation energy.
The single-particle basis functions  could be the results of a Hartree-Fock calculation or just the eigenstates of the non-interacting part of the Hamiltonian. 

In our notes on Hartree-Fock calculations, 
we have already computed the matrix $\langle \Phi_0 | \hat{H}|\Phi_{i}^{a}\rangle $ and $\langle \Phi_0 | \hat{H}|\Phi_{ij}^{ab}\rangle$.  If we are using a Hartree-Fock basis, then the matrix elements
$\langle \Phi_0 | \hat{H}|\Phi_{i}^{a}\rangle=0$ and we are left with a \emph{correlation energy} given by
\[
E-E_0 =\Delta E^{HF}=\sum_{abij}\langle \Phi_0 | \hat{H}|\Phi_{ij}^{ab} \rangle C_{ij}^{ab}. 
\]


Inserting the various matrix elements we can rewrite the previous equation as
\[
\Delta E=\sum_{ai}\langle i| \hat{f}|a \rangle C_{i}^{a}+
\sum_{abij}\langle ij | \hat{v}| ab \rangle C_{ij}^{ab}.
\]
This equation determines the correlation energy but not the coefficients $C$. 
We need more equations. Our next step is to set up
\[
\langle \Phi_i^a | \hat{H} -E| \Phi_0\rangle + \sum_{bj}\langle \Phi_i^a | \hat{H} -E|\Phi_{j}^{b} \rangle C_{j}^{b}+
\sum_{bcjk}\langle \Phi_i^a | \hat{H} -E|\Phi_{jk}^{bc} \rangle C_{jk}^{bc}+
\sum_{bcdjkl}\langle \Phi_i^a | \hat{H} -E|\Phi_{jkl}^{bcd} \rangle C_{jkl}^{bcd}=0,
\]
as this equation will allow us to find an expression for the coefficents $C_i^a$ since we can rewrite this equation as 
\[
\langle i | \hat{f}| a\rangle +\langle \Phi_i^a | \hat{H}|\Phi_{i}^{a} \rangle C_{i}^{a}+ \sum_{bj\ne ai}\langle \Phi_i^a | \hat{H}|\Phi_{j}^{b} \rangle C_{j}^{b}+
\sum_{bcjk}\langle \Phi_i^a | \hat{H}|\Phi_{jk}^{bc} \rangle C_{jk}^{bc}+
\sum_{bcdjkl}\langle \Phi_i^a | \hat{H}|\Phi_{jkl}^{bcd} \rangle C_{jkl}^{bcd}=EC_i^a.
\]

We see that on the right-hand side we have the energy $E$. This leads to a non-linear equation in the unknown coefficients. 
These equations are normally solved iteratively ( that is we can start with a guess for the coefficients $C_i^a$). A common choice is to use perturbation theory for the first guess, setting thereby
\[
 C_{i}^{a}=\frac{\langle i | \hat{f}| a\rangle}{\epsilon_i-\epsilon_a}.
\]

The observant reader will however see that we need an equation for $C_{jk}^{bc}$ and $C_{jkl}^{bcd}$ as well.
To find equations for these coefficients we need then to continue our multiplications from the left with the various
$\Phi_{H}^P$ terms. 


For $C_{jk}^{bc}$ we need then
\[
\langle \Phi_{ij}^{ab} | \hat{H} -E| \Phi_0\rangle + \sum_{kc}\langle \Phi_{ij}^{ab} | \hat{H} -E|\Phi_{k}^{c} \rangle C_{k}^{c}+
\]
\[
\sum_{cdkl}\langle \Phi_{ij}^{ab} | \hat{H} -E|\Phi_{kl}^{cd} \rangle C_{kl}^{cd}+\sum_{cdeklm}\langle \Phi_{ij}^{ab} | \hat{H} -E|\Phi_{klm}^{cde} \rangle C_{klm}^{cde}+\sum_{cdefklmn}\langle \Phi_{ij}^{ab} | \hat{H} -E|\Phi_{klmn}^{cdef} \rangle C_{klmn}^{cdef}=0,
\]
and we can isolate the coefficients $C_{kl}^{cd}$ in a similar way as we did for the coefficients $C_{i}^{a}$. 
A standard choice for the first iteration is to set 
\[
C_{ij}^{ab} =\frac{\langle ij \vert \hat{v} \vert ab \rangle}{\epsilon_i+\epsilon_j-\epsilon_a-\epsilon_b}.
\]
At the end we can rewrite our solution of the Schroedinger equation in terms of $n$ coupled equations for the coefficients $C_H^P$.
This is a very cumbersome way of solving the equation. However, by using this iterative scheme we can illustrate how we can compute the
various terms in the wave operator or correlation operator $\hat{C}$. We will later identify the calculation of the various terms $C_H^P$
as parts of different many-body approximations to full CI. In particular, we can  relate this non-linear scheme with Coupled Cluster theory and
many-body perturbation theory.


\subsection*{Summarizing FCI and bringing in approximative methods}


If we can diagonalize large matrices, FCI is the method of choice since:
\begin{itemize}
\item It gives all eigenvalues, ground state and excited states

\item The eigenvectors are obtained directly from the coefficients $C_H^P$ which result from the diagonalization

\item We can compute easily expectation values of other operators, as well as transition probabilities

\item Correlations are easy to understand in terms of contributions to a given operator beyond the Hartree-Fock contribution. This is the standard approach in  many-body theory. 
\end{itemize}

\noindent
The correlation energy is defined as, with a two-body Hamiltonian,  
\[
\Delta E=\sum_{ai}\langle i| \hat{f}|a \rangle C_{i}^{a}+
\sum_{abij}\langle ij | \hat{v}| ab \rangle C_{ij}^{ab}.
\]

The coefficients $C$ result from the solution of the eigenvalue problem. 
The energy of say the ground state is then
\[
E=E_{ref}+\Delta E,
\]
where the so-called reference energy is the energy we obtain from a Hartree-Fock calculation, that is
\[
E_{ref}=\langle \Phi_0 \vert \hat{H} \vert \Phi_0 \rangle.
\]

However, as we have seen, even for a small case like the four first major shells and a nucleus like oxygen-16, the dimensionality becomes quickly intractable. If we wish to include single-particle states that reflect weakly bound systems, we need a much larger single-particle basis. We need thus approximative methods that sum specific correlations to infinite order. 

Popular methods are
\begin{itemize}
\item Many-body perturbation theory (in essence a Taylor expansion)

\item Coupled cluster theory (coupled non-linear equations)

\item Green's function approaches (matrix inversion)

\item Similarity group transformation methods (coupled ordinary differential equations
\end{itemize}

\noindent
All these methods start normally with a Hartree-Fock basis as the calculational basis. 

\paragraph{Many-body perturbation theory, in short.}
If we are looking at the ground state energy only, we define an effective Hilbert space which comprises only this state. This space is called the \textbf{model space}.

We can express the correlation energy as a perturbative expression in terms
of $\hat{H}_I$ 
\[
\Delta E=\sum_{i=1}^{\infty}\Delta E^{(i)}.
\]
We get the following expression for $\Delta E^{(i)}$
\[
\Delta E^{(1)}=\langle \Phi_0\vert \hat{H}_I\vert \Phi_0\rangle,
\] 
which is just the contribution to first order in perturbation theory,
\[
\Delta E^{(2)}=\langle\Phi_0\vert \hat{H}_I\frac{\hat{Q}}{W_0-\hat{H}_0}\hat{H}_I\vert \Phi_0\rangle, 
\]
which is the contribution to second order.



\[
\Delta E^{(3)}=\langle \Phi_0\vert \hat{H}_I\frac{\hat{Q}}{W_0-\hat{H}_0}\hat{H}_I\frac{\hat{Q}}{W_0-\hat{H}_0}\hat{H}_I\Phi_0\rangle-
\langle\Phi_0\vert \hat{H}_I\frac{\hat{Q}}{W_0-\hat{H}_0}\langle \Phi_0\vert \hat{H}_I\vert \Phi_0\rangle\frac{\hat{Q}}{W_0-\hat{H}_0}\hat{H}_I\vert \Phi_0\rangle,
\]
being the third-order contribution. 

In the shell-model we showed that we could rewrite the exact state function for say the ground state, as a linear expansion in terms of all possible Slater determinants. That is, we 
define the ansatz for the ground state as 
\[
|\Phi_0\rangle = \left(\prod_{i\le F}\hat{a}_{i}^{\dagger}\right)|0\rangle,
\]
where the index $i$ defines different single-particle states up to the Fermi level. We have assumed that we have $N$ fermions. 
A given one-particle-one-hole ($1p1h$) state can be written as
\[
|\Phi_i^a\rangle = \hat{a}_{a}^{\dagger}\hat{a}_i|\Phi_0\rangle,
\]
while a $2p2h$ state can be written as
\[
|\Phi_{ij}^{ab}\rangle = \hat{a}_{a}^{\dagger}\hat{a}_{b}^{\dagger}\hat{a}_j\hat{a}_i|\Phi_0\rangle,
\]
and a general $ApAh$ state as 
\[
|\Phi_{ijk\dots}^{abc\dots}\rangle = \hat{a}_{a}^{\dagger}\hat{a}_{b}^{\dagger}\hat{a}_{c}^{\dagger}\dots\hat{a}_k\hat{a}_j\hat{a}_i|\Phi_0\rangle.
\]

We use letters $ijkl\dots$ for states below the Fermi level and $abcd\dots$ for states above the Fermi level. A general single-particle state is given by letters $pqrs\dots$.

We can then expand our exact state function for the ground state 
as
\[
|\Psi_0\rangle=C_0|\Phi_0\rangle+\sum_{ai}C_i^a|\Phi_i^a\rangle+\sum_{abij}C_{ij}^{ab}|\Phi_{ij}^{ab}\rangle+\dots
=(C_0+\hat{C})|\Phi_0\rangle,
\]
where we have introduced the so-called correlation operator 
\[
\hat{C}=\sum_{ai}C_i^a\hat{a}_{a}^{\dagger}\hat{a}_i  +\sum_{abij}C_{ij}^{ab}\hat{a}_{a}^{\dagger}\hat{a}_{b}^{\dagger}\hat{a}_j\hat{a}_i+\dots
\]
Since the normalization of $\Psi_0$ is at our disposal and since $C_0$ is by hypothesis non-zero, we may arbitrarily set $C_0=1$ with 
corresponding proportional changes in all other coefficients. Using this so-called intermediate normalization we have
\[
\langle \Psi_0 | \Phi_0 \rangle = \langle \Phi_0 | \Phi_0 \rangle = 1, 
\]
resulting in 
\[
|\Psi_0\rangle=(1+\hat{C})|\Phi_0\rangle.
\]

In a shell-model calculation, the unknown coefficients in $\hat{C}$ are the 
eigenvectors which result from the diagonalization of the Hamiltonian matrix.

How can we use perturbation theory to determine the same coefficients? Let us study the contributions to second order in the interaction, namely
\[
\Delta E^{(2)}=\langle\Phi_0\vert \hat{H}_I\frac{\hat{Q}}{W_0-\hat{H}_0}\hat{H}_I\vert \Phi_0\rangle.
\]

The intermediate states given by $\hat{Q}$ can at most be of a $2p-2h$ nature if we have a two-body Hamiltonian. This means that second order in the perturbation theory can have $1p-1h$ and $2p-2h$ at most as intermediate states. When we diagonalize, these contributions are included to infinite order. This means that higher-orders in perturbation theory bring in more complicated correlations. 

If we limit the attention to a Hartree-Fock basis, then we have that
$\langle\Phi_0\vert \hat{H}_I \vert 2p-2h\rangle$ is the only contribution and the contribution to the energy reduces to
\[
\Delta E^{(2)}=\frac{1}{4}\sum_{abij}\langle ij\vert \hat{v}\vert ab\rangle \frac{\langle ab\vert \hat{v}\vert ij\rangle}{\epsilon_i+\epsilon_j-\epsilon_a-\epsilon_b}.
\]

If we compare this to the correlation energy obtained from full configuration interaction theory with a Hartree-Fock basis, we found that
\[
E-E_0 =\Delta E=\sum_{abij}\langle ij | \hat{v}| ab \rangle C_{ij}^{ab},
\]
where the energy $E_0$ is the reference energy and $\Delta E$ defines the so-called correlation energy.

We see that if we set
\[
C_{ij}^{ab} =\frac{1}{4}\frac{\langle ab \vert \hat{v} \vert ij \rangle}{\epsilon_i+\epsilon_j-\epsilon_a-\epsilon_b},
\]
we have a perfect agreement between FCI and MBPT. However, FCI includes such $2p-2h$ correlations to infinite order. In order to make a meaningful comparison we would at least need to sum such correlations to infinite order in perturbation theory. 

Summing up, we can see that
\begin{itemize}
\item MBPT introduces order-by-order specific correlations and we make comparisons with exact calculations like FCI

\item At every order, we can calculate all contributions since they are well-known and either tabulated or calculated on the fly.

\item MBPT is a non-variational theory and there is no guarantee that higher orders will improve the convergence. 

\item However, since FCI calculations are limited by the size of the Hamiltonian matrices to diagonalize (today's most efficient codes can attach dimensionalities of ten billion basis states, MBPT can function as an approximative method which gives a straightforward (but tedious) calculation recipe. 

\item MBPT has been widely used to compute effective interactions for the nuclear shell-model.

\item But there are better methods which sum to infinite order important correlations. Coupled cluster theory is one of these methods. 
\end{itemize}

\noindent
\paragraph{A quick tour of Coupled Cluster theory.}
The ansatz for the wavefunction (ground state) is given by
\begin{equation*}
   \vert \Psi\rangle = \vert \Psi_{CC}\rangle = e^{\hat{T}} \vert \Phi_0\rangle =  
  \left( \sum_{n=1}^{A} \frac{1}{n!} \hat{T}^n \right) \vert \Phi_0\rangle,
\end{equation*}
where $A$ represents the maximum number of particle-hole excitations and $\hat{T}$ is the cluster operator defined as
\begin{align*}
            \hat{T} &= \hat{T}_1 + \hat{T}_2 + \ldots + \hat{T}_A \\
            \hat{T}_n &= \left(\frac{1}{n!}\right)^2 
                \sum_{\substack{
                        i_1,i_2,\ldots i_n \\
                        a_1,a_2,\ldots a_n}}
                t_{i_1i_2\ldots i_n}^{a_1a_2\ldots a_n} a_{a_1}^\dagger a_{a_2}^\dagger \ldots a_{a_n}^\dagger a_{i_n} \ldots a_{i_2} a_{i_1}.
        \end{align*}
    The energy is given by
    \begin{equation*}
        E_{\mathrm{CC}} = \langle\Phi_0\vert  \overline{H}\vert \Phi_0\rangle,
    \end{equation*}
    where $\overline{H}$ is a similarity transformed Hamiltonian
    \begin{align*}
        \overline{H}&= e^{-\hat{T}} \hat{H}_N e^{\hat{T}} \\
        \hat{H}_N &= \hat{H} - \langle\Phi_0\vert \hat{H} \vert \Phi_0\rangle.
    \end{align*}

    The coupled cluster energy is a function of the unknown cluster amplitudes $t_{i_1i_2\ldots i_n}^{a_1a_2\ldots a_n}$,
given by the solutions to the amplitude equations
    \begin{equation*}
        0 = \langle\Phi_{i_1 \ldots i_n}^{a_1 \ldots a_n}\vert \overline{H}\vert \Phi_0\rangle.
    \end{equation*}
The similarity transformed   Hamiltonian  $\overline{H}$ is expanded using the Baker-Campbell-Hausdorff expression,
    \begin{align*}
        \overline{H}&= \hat{H}_N + \left[ \hat{H}_N, \hat{T} \right] + 
            \frac{1}{2} \left[\left[ \hat{H}_N, \hat{T} \right], \hat{T}\right] + \ldots \\
            & \quad \frac{1}{n!} \left[ \ldots \left[ \hat{H}_N, \hat{T} \right], \ldots \hat{T} \right] +\dots
    \end{align*}
and simplified using the connected cluster theorem
    \begin{equation*}
        \overline{H}= \hat{H}_N + \left( \hat{H}_N \hat{T}\right)_c + \frac{1}{2} \left( \hat{H}_N \hat{T}^2\right)_c
            + \dots + \frac{1}{n!} \left( \hat{H}_N \hat{T}^n\right)_c +\dots
    \end{equation*}

A much used approximation is to  truncate the cluster operator $\hat{T}$ at the $n=2$ level. This defines the so-called approximation to the Coupled Cluster wavefunction.

The coupled cluster wavefunction is now given by
\begin{equation*}
            \vert \Psi_{CC}\rangle = e^{\hat{T}_1 + \hat{T}_2} \vert \Phi_0\rangle
\end{equation*}
where 
        \begin{align*}
            \hat{T}_1 &= 
            \sum_{ia}
                t_{i}^{a} a_{a}^\dagger a_i \\
            \hat{T}_2 &= \frac{1}{4} 
            \sum_{ijab}
                t_{ij}^{ab} a_{a}^\dagger a_{b}^\dagger a_{j} a_{i}.
        \end{align*}

The amplutudes $t$ play a role similar to the coefficients $C$ in the shell-model calculations. They are obtained by solving a set of non-linear equations
similar to those discussed above in connection withe FCI discussion.

If we truncate our equations at the CCSD level, it corresponds to performing a transformation of the Hamiltonian matrix of the following type for a six particle problem (with a two-body Hamiltonian):


\begin{quote}
\begin{tabular}{cccccccc}
\hline
\multicolumn{1}{c}{  } & \multicolumn{1}{c}{ $0p-0h$ } & \multicolumn{1}{c}{ $1p-1h$ } & \multicolumn{1}{c}{ $2p-2h$ } & \multicolumn{1}{c}{ $3p-3h$ } & \multicolumn{1}{c}{ $4p-4h$ } & \multicolumn{1}{c}{ $5p-5h$ } & \multicolumn{1}{c}{ $6p-6h$ } \\
\hline
$0p-0h$ & $\tilde{x}$ & $\tilde{x}$ & $\tilde{x}$ & 0           & 0           & 0           & 0           \\
$1p-1h$ & 0           & $\tilde{x}$ & $\tilde{x}$ & $\tilde{x}$ & 0           & 0           & 0           \\
$2p-2h$ & 0           & $\tilde{x}$ & $\tilde{x}$ & $\tilde{x}$ & $\tilde{x}$ & 0           & 0           \\
$3p-3h$ & 0           & $\tilde{x}$ & $\tilde{x}$ & $\tilde{x}$ & $\tilde{x}$ & $\tilde{x}$ & 0           \\
$4p-4h$ & 0           & 0           & $\tilde{x}$ & $\tilde{x}$ & $\tilde{x}$ & $\tilde{x}$ & $\tilde{x}$ \\
$5p-5h$ & 0           & 0           & 0           & $\tilde{x}$ & $\tilde{x}$ & $\tilde{x}$ & $\tilde{x}$ \\
$6p-6h$ & 0           & 0           & 0           & 0           & $\tilde{x}$ & $\tilde{x}$ & $\tilde{x}$ \\
\hline
\end{tabular}
\end{quote}

\noindent


In our FCI discussion the correlation energy is defined as, with a two-body Hamiltonian,  
\[
\Delta E=\sum_{ai}\langle i| \hat{f}|a \rangle C_{i}^{a}+
\sum_{abij}\langle ij | \hat{v}| ab \rangle C_{ij}^{ab}.
\]

In Coupled cluster theory it becomes (irrespective of level of truncation of $T$)
\[
\Delta E=\sum_{ai}\langle i| \hat{f}|a \rangle t_{i}^{a}+
\sum_{abij}\langle ij | \hat{v}| ab \rangle t_{ij}^{ab}.
\]

Coupled cluster theory has several interesting computational features and is the method of choice in quantum chemistry. There are several interesting features:
\begin{itemize}
\item With a truncation like CCSD or CCSDT, we can include to infinite order correlations like $2p-2h$.

\item We can include a large basis of single-particle states, not possible in standard FCI calculations
\end{itemize}

\noindent
However, Coupled Cluster theory is
\begin{itemize}
\item non-variational

\item if we want to find properties of excited states, additional calculations via for example equation of motion methods are needed

\item if correlations are strong, a single-reference ansatz may not be the best starting point

\item we cannot quantify properly the error we make when truncations are made in the cluster operator
\end{itemize}

\noindent

% ------------------- end of main content ---------------


\printindex

\end{document}

